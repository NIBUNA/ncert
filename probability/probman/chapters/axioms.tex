\subsection{Boolean Logic}
If A and B are two events such that P(A) = $\frac{1}{4}$, P(B) = $\frac{1}{2}$ and P(A $\cap$ B) = $\frac{1}{8}$. find P (not A and not B).
\renewcommand{\theequation}{\theenumi}
\begin{enumerate}[label=\thesubsection.\arabic*.,ref=\thesubsection.\theenumi]
\numberwithin{equation}{enumi}

\item 
\begin{align}
A^{\prime}B^{\prime} &=  \brak{A+B}^{\prime}
\\
\implies \pr{A^{\prime}B^{\prime}} &=  \pr{\brak{A+B}^{\prime}} 
\\
&= 1 - \pr{A+B} 
\label{eq:axiom_sum_one}
\end{align}
\item 
\begin{align}
\because A+B &= A\brak{B+B^{\prime}} + B
\\
&= B \brak{A +1} + A B^{\prime}
\\
&=B + A B^{\prime}
\\
\implies \pr{A+B} &= \pr{B + A B^{\prime} }
\\
&=\pr{B}+\pr{ A B^{\prime} } 
\\
&\because B \brak{ A B^{\prime} } = 0
\label{eq:axiom_sum_two}
\end{align}
\item 
\begin{align}
A = A \brak{B+B^{\prime}} =  AB + AB^{\prime}
\label{eq:axiom_sum_A}
\end{align}
and 
\begin{align}
\brak{ AB}\brak{  AB^{\prime}} = 0, \because BB^{\prime} = 0
\label{eq:axiom_sum_AB0}
\end{align}
Hence, $AB$ and $AB^{\prime}$ are mutually exclusive and 
%
\begin{align}
\pr{A} = \pr{AB} + \pr{AB^{\prime}}
\\
\implies 
\pr{AB^{\prime}} =  \pr{A} - \pr{AB}
\label{eq:axiom_sum_ABp}
\end{align}
\item Substituting \eqref{eq:axiom_sum_ABp} in \eqref{eq:axiom_sum_two}, 
\begin{align}
\pr{A+B} &= \pr{A} + \pr{B} - \pr{AB} 
\label{eq:axiom_sum_AB}
\end{align}
\item Substituting \eqref{eq:axiom_sum_AB} in \eqref{eq:axiom_sum_one}
\begin{align}
\pr{A^{\prime}B^{\prime}} &=  1 - \cbrak{\pr{A} + \pr{B} - \pr{AB} }
\\
&= 1 - \brak{\frac{1}{4} + \frac{1}{2} - \frac{1}{8}}
\\
&= \frac{3}{8}
\label{eq:axiom_sum_final}
\end{align}
\end{enumerate}
\subsection{Independence}
\renewcommand{\theequation}{\theenumi}
\begin{enumerate}[label=\thesubsection.\arabic*.,ref=\thesubsection.\theenumi]
\numberwithin{equation}{enumi}



\item Prove that if $E$ and $F$ are independent events, then so are the events $E$ and $F^{\prime}$.\\
\solution  If $E$ and $F$ are independent,
\begin{align}
\pr{EF} = \pr{E}\pr{F}
\label{eq:axiom_indep}
\end{align}
Using Boolean algebra,
\begin{align}
E = E \brak{F+F^{\prime}} =  EF + EF^{\prime}
\label{eq:axiom_indep_E}
\end{align}
and 
\begin{align}
\brak{ EF}\brak{  EF^{\prime}} = 0, \because FF^{\prime} = 0
\label{eq:axiom_indep_EF0}
\end{align}
Hence, $EF$ and $EF^{\prime}$ are mutually exclusive and 
%
\begin{align}
\pr{E} = \pr{EF} + \pr{EF^{\prime}}
\\
\implies 
\pr{EF^{\prime}} =  \pr{E} - \pr{EF}
\label{eq:axiom_indep_EFp}
\end{align}
Substituting from \eqref{eq:axiom_indep} in \eqref{eq:axiom_indep_EFp},
%
\begin{align}
\pr{EF^{\prime}} &=  \pr{E} \brak{1- \pr{F}}
&= \pr{E} \pr{F^{\prime}}
\label{eq:axiom_indep_EFp_ind}
\end{align}
%
\begin{align}
\because FF^{\prime} = 0, F + F^{\prime} = 1
\\
\implies \pr{F}+\pr{F^{\prime}} = 1
\label{eq:axiom_FFp}
\end{align}
By definition, from \eqref{eq:axiom_indep_EFp_ind}, we conclude that $E$ and $F^{\prime}$ are independent.
\item If A and B are two independent events, then the probability of occurrence of at least one of A and B is given by 1- $P(A^{\prime}) P(B^{\prime})$\\
\solution 
\begin{align}
\because (A+B)(A+B)^{\prime} = 0
\\
\implies 1 = \pr{A+B} + \pr{\brak{A+B}^{\prime}}
\\
\implies \pr{A+B} = 1 - \pr{A^{\prime}B^{\prime}} 
\\
= 1 - \pr{A^{\prime}}\pr{B^{\prime}} 
\end{align}
using the definition of independence.
%\input{./solutions/20-30/chapters/prob/examples/docq24.tex}
\end{enumerate}
\subsection{Conditional Probability}
\renewcommand{\theequation}{\theenumi}
\begin{enumerate}[label=\thesubsection.\arabic*.,ref=\thesubsection.\theenumi]
\numberwithin{equation}{enumi}

\item Given that E and F are events such that P(E) = 0.6, P(F) = 0.3 and P(E $\cap$ F) = 0.2, find P(E/F) and P(F/E)?\\
\item Two events A and B will be independent, if
\begin{enumerate}
\item A and B are mutually exclusive
\item $P(A^{\prime}B^{\prime})$ = [1 – P(A)] [1 – P(B)]
\item P(A) = P(B)
\item P(A) + P(B) = 1
\end{enumerate}
\solution
%Let $X_i \in \cbrak{0,1}$ represent the $ith$ hurdle where $1$ denotes a hurdle being knocked down and let
\begin{align}
X = \sum_{i=1}^{n}X_i
\end{align}
%
where $n$ is the total number of hurdles.
Then $X$ has a binomial distribution with
%
\begin{align}
\pr{X = k} = \comb{n}{k}p^{n-k}\brak{1-p}^k
\end{align}
%
where 
\begin{align}
p &= \frac{5}{6} = \frac{1}{6}
\\
n &= 10
\end{align}
from the given information.  Then,
\begin {align}
\pr{X < 2} = \pr{X = 0} + \pr{X=1}
\\
= \comb{10}{0}\brak{\frac{5}{6}}^{10}\brak{\frac{1}{6}}^0+ 
 \comb{10}{1}\brak{\frac{5}{6}}^{9}\brak{\frac{1}{6}}^1 
\\
=\brak{\frac{5}{6}}^{9}\cbrak{\frac{5}{6}+\frac{10}{6}}
\\
=3\brak{\frac{5}{6}}^{10}
\end{align}
%
which is the desired probability.

\item If A and B are events such that P(A/B) = P(B/A), then
\begin{enumerate}
\item A $\subset$ B but A $\neq$ B
\item A = B
\item A $\cap$ B = $\phi$
\item P(A) = P(B)
\end{enumerate}
\solution
%\input{./solutions/20-30/chapters/prob/exercises/docq28.tex}

\item Which of the following cannot be the probability of an event?\\
(A)$\frac{2}{3}$(B) –1.5 (C) 15 (D) 0.7
\item If P(E) = 0.05, what is the probability of ‘not E’?
\item If A and B are two events such that P(A) $\neq$ 0 and P(B/A) = 1, then
(A) A $\subset$ B \\
(B) B $\subset$ A \\
(C) B = $\phi$ \\
(D) A = $\phi$\\

\item If P(A/B) $>$ P(A), then which of the following is correct :
(A) P(B/A) $<$ P(B) \\
(B) P(A $\cap$ B) $<$ P(A) . P(B)\\
(C) P(B/A) $>$ P(B) \\
(D) P(B/A) = P(B)

\item If A and B are any two events such that P(A) + P(B) – P(A and B) = P(A), then\\
(A) P(B/A) = 1 \\
(B) P(A/B) = 1\\
(C) P(B/A) = 0 \\
(D) P(A/B) = 0\\
\item Complete the following statements:\\
 (i) Probability of an event E + Probability of the event ‘not E’ =----------- .\\
 (ii) The probability of an event that cannot happen is---------- . Such an event is called--------- .\\
 (iii) The probability of an event that is certain to happen is ---------.\\
 (iv) The sum of the probabilities of all the elementary events of an experiment is----------.\\ (v) The probability of an event is greater than or equal to and less than or equal to --------------.\\\item An electronic assembly consists of two subsystems, say, A and B. From previous testing procedures, the following probabilities are assumed to be known:\\
P(A fails) = 0.2\\
P(B fails alone) = 0.15\\
P(A and B fail) = 0.15\\
\\Evaluate the following probabilities\\
(i) P(A fails|B has failed) \\
(ii) P(A fails alone)\\
\item A and B are two events such that P (A) $\neq$ 0. Find P(B/A), if\\
(i) A is a subset of B \\
(ii) A $\cap$ B = $\phi$\\
\item If A and B are two events such that A $\subset$ B and P(B) $\neq$ 0, then which of the following is correct?\\
\begin{enumerate}
\item P(A/B) = $\frac{P(B)}{P(A)}$
\item $P(A/B) < P(A)$
\item P(A/B) $\geq$ P(A)
\item None of these
\end{enumerate}

\item Given that the events A and B are such that P(A) = $\frac{1}{2}$, P(A $\cup B$)= $\frac{3}{5}$ and P(B) = p. Find p if they are\\
(i) mutually exclusive\\
(ii) independent.\\

\item Let A and B be independent events with P(A) = 0.3 and P(B) = 0.4. Find\\
(i) P(A $\cap$ B)\\ 
(ii) P(A $\cup$ B)\\
(iii) P(A/B)\\
(iv) P(B/A)\\


\item Events A and B are such that P (A) = $\frac{1}{2}$, P(B) = $\frac{7}{12}$ and P(not A or not B) = $\frac{1}{4}$. State whether A and B are independent ?\\

\item Given two independent events A and B such that P(A) = 0.3, P(B) = 0.6. Find\\
(i) P(A and B)\\
(ii) P(A and not B)\\
(iii) P(A or B)\\
(iv) P(neither A nor B)\\
\item A die marked 1, 2, 3 in red and 4, 5, 6 in green is tossed. Let A be the event, 'the number is even,' and B be the event, 'the number is red'. Are A and B
independent?\\
\item A person plays a game of tossing a coin thrice. For each head, he is given Rs 2 by the organiser of the game and for each tail, he has to give Rs 1.50 to the organiser. Let X denote the amount gained or lost by the person. Show that X is a random variable and exhibit it as a function on the sample space of the experiment.\\

\item If P(A)=$\frac{7}{13}, P(B)=\frac{9}{13}$ and $P(A\cap B)=\frac{4}{13},$ Evaluate P(A/B)?
\\

\item A die is thrown. If E is the event "the number appearing is a multiple of 3" and F be the event "the number appearing is even" then find whether E and F are independent ?\\

\item An unbiased die is thrown twice. Let the event A be "odd number on the first throw" and B the event "odd number on the second throw". Check the independence of the events A and B.\\
%\solution
%\input{./solutions/20-30/chapters/prob/examples/docq21.tex}



\item Compute P(A/B), if P(B) = 0.5 and P (A $\cap$ B) = 0.32.\\

\item If P(A) = 0.8, P(B) = 0.5 and P(B/A) = 0.4, find\\
(i) P(A $\cap$ B)\\
(ii) P(A/B)\\ 
(iii) P(A $\cup$ B)\\

\item Evaluate P(A $\cup$ B), if 2P(A) = P(B)  = $\frac{5}{13}$ and P(A/B) =  $\frac{2}{5}.$\\

\item If P(A) = $\frac{6}{11}$, P(B) = $\frac{5}{11}$ and P(A $\cup$ B) = $\frac{11}{7}$ find\\
(i) P(A $\cap$ B)\\ 
(ii) P(A/B)\\ 
(iii) P(B/A)
\\
\item A fair die is rolled. Consider the events E =  (1, 3, 5), F = (2, 3) and G = (2, 3, 4, 5) Find\\
(i) P(E/F) and P(F/E) \\
(ii) P(E/G) and P(G/E)\\
(iii) P((E $\cup$ F)/G) and P((E $\cap$ F)/G)\\
%\solution
%\input{./solutions/20-30/chapters/prob/exercises/docq22.tex}
\item Choose the correct answer, if P(A) = $\frac{1}{2}$, P(B) = 0, then P(A/B) is
\begin{enumerate}
\item 0
\item $\frac{1}{2}$
\item not defined
\item 1
\end{enumerate}
%\solution
%\input{./solutions/20-30/chapters/prob/exercises/docq27.tex}

\item Let E and F be events with P(E) = $\frac{3}{5}$, P(F) = $\frac{3}{10}$ and  P(E $\cap$ F) = $\frac{1}{5}$. Are E and F independent?\\

\item One card is drawn at random from a well shuffled deck of 52 cards. In which of the following cases are the events E and F independent?\\
(i) E : 'the card drawn is a spade'
F : 'the card drawn is an ace'\\
(ii) E : 'the card drawn is black'
F : 'the card drawn is a king'\\
(iii) E : 'the card drawn is a king or queen'
F : 'the card drawn is a queen or jack'.\\
%\solution
%Let $X_i \in \cbrak{0,1}$ represent the $ith$ hurdle where $1$ denotes a hurdle being knocked down and let
\begin{align}
X = \sum_{i=1}^{n}X_i
\end{align}
%
where $n$ is the total number of hurdles.
Then $X$ has a binomial distribution with
%
\begin{align}
\pr{X = k} = \comb{n}{k}p^{n-k}\brak{1-p}^k
\end{align}
%
where 
\begin{align}
p &= \frac{5}{6} = \frac{1}{6}
\\
n &= 10
\end{align}
from the given information.  Then,
\begin {align}
\pr{X < 2} = \pr{X = 0} + \pr{X=1}
\\
= \comb{10}{0}\brak{\frac{5}{6}}^{10}\brak{\frac{1}{6}}^0+ 
 \comb{10}{1}\brak{\frac{5}{6}}^{9}\brak{\frac{1}{6}}^1 
\\
=\brak{\frac{5}{6}}^{9}\cbrak{\frac{5}{6}+\frac{10}{6}}
\\
=3\brak{\frac{5}{6}}^{10}
\end{align}
%
which is the desired probability.

\end{enumerate}

