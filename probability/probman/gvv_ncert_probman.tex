\documentclass[journal,12pt,twocolumn]{IEEEtran}
%
\usepackage{setspace}
\usepackage{gensymb}
%\doublespacing
\singlespacing

%\usepackage{graphicx}
%\usepackage{amssymb}
%\usepackage{relsize}
\usepackage[cmex10]{amsmath}
\usepackage{siunitx}
%\usepackage{amsthm}
%\interdisplaylinepenalty=2500
%\savesymbol{iint}
%\usepackage{txfonts}
%\restoresymbol{TXF}{iint}
%\usepackage{wasysym}
\usepackage{amsthm}
\usepackage{iithtlc}
\usepackage{mathrsfs}
\usepackage{txfonts}
\usepackage{stfloats}
\usepackage{steinmetz}
%\usepackage{bm}
\usepackage{cite}
\usepackage{cases}
\usepackage{subfig}
%\usepackage{xtab}
\usepackage{longtable}
\usepackage{multirow}
%\usepackage{algorithm}
%\usepackage{algpseudocode}
\usepackage{enumitem}
\usepackage{mathtools}
\usepackage{tikz}
\usepackage{circuitikz}
\usepackage{verbatim}
\usepackage{tfrupee}
\usepackage[breaklinks=true]{hyperref}
%\usepackage{stmaryrd}
\usepackage{tkz-euclide} % loads  TikZ and tkz-base
%\usetkzobj{all}
\usetikzlibrary{calc,math}
\usetikzlibrary{fadings}
\usepackage{listings}
    \usepackage{color}                                            %%
    \usepackage{array}                                            %%
    \usepackage{longtable}                                        %%
    \usepackage{calc}                                             %%
    \usepackage{multirow}                                         %%
    \usepackage{hhline}                                           %%
    \usepackage{ifthen}                                           %%
  %optionally (for landscape tables embedded in another document): %%
    \usepackage{lscape}     
\usepackage{multicol}
\usepackage{chngcntr}
%\usepackage{enumerate}

%\usepackage{wasysym}
%\newcounter{MYtempeqncnt}
\DeclareMathOperator*{\Res}{Res}
%\renewcommand{\baselinestretch}{2}
\renewcommand\thesection{\arabic{section}}
\renewcommand\thesubsection{\thesection.\arabic{subsection}}
\renewcommand\thesubsubsection{\thesubsection.\arabic{subsubsection}}

\renewcommand\thesectiondis{\arabic{section}}
\renewcommand\thesubsectiondis{\thesectiondis.\arabic{subsection}}
\renewcommand\thesubsubsectiondis{\thesubsectiondis.\arabic{subsubsection}}

% correct bad hyphenation here
\hyphenation{op-tical net-works semi-conduc-tor}
\def\inputGnumericTable{}                                 %%

\lstset{
%language=C,
frame=single, 
breaklines=true,
columns=fullflexible
}
%\lstset{
%language=tex,
%frame=single, 
%breaklines=true
%}

\begin{document}
%


\newtheorem{theorem}{Theorem}[section]
\newtheorem{problem}{Problem}
\newtheorem{proposition}{Proposition}[section]
\newtheorem{lemma}{Lemma}[section]
\newtheorem{corollary}[theorem]{Corollary}
\newtheorem{example}{Example}[section]
\newtheorem{definition}[problem]{Definition}
%\newtheorem{thm}{Theorem}[section] 
%\newtheorem{defn}[thm]{Definition}
%\newtheorem{algorithm}{Algorithm}[section]
%\newtheorem{cor}{Corollary}
\newcommand{\BEQA}{\begin{eqnarray}}
\newcommand{\EEQA}{\end{eqnarray}}
\newcommand{\define}{\stackrel{\triangle}{=}}

\bibliographystyle{IEEEtran}
%\bibliographystyle{ieeetr}


\providecommand{\mbf}{\mathbf}
\providecommand{\pr}[1]{\ensuremath{\Pr\left(#1\right)}}
\providecommand{\qfunc}[1]{\ensuremath{Q\left(#1\right)}}
\providecommand{\sbrak}[1]{\ensuremath{{}\left[#1\right]}}
\providecommand{\lsbrak}[1]{\ensuremath{{}\left[#1\right.}}
\providecommand{\rsbrak}[1]{\ensuremath{{}\left.#1\right]}}
\providecommand{\brak}[1]{\ensuremath{\left(#1\right)}}
\providecommand{\lbrak}[1]{\ensuremath{\left(#1\right.}}
\providecommand{\rbrak}[1]{\ensuremath{\left.#1\right)}}
\providecommand{\cbrak}[1]{\ensuremath{\left\{#1\right\}}}
\providecommand{\lcbrak}[1]{\ensuremath{\left\{#1\right.}}
\providecommand{\rcbrak}[1]{\ensuremath{\left.#1\right\}}}
\theoremstyle{remark}
\newtheorem{rem}{Remark}
\newcommand{\sgn}{\mathop{\mathrm{sgn}}}
\providecommand{\abs}[1]{\left\vert#1\right\vert}
\providecommand{\res}[1]{\Res\displaylimits_{#1}} 
\providecommand{\norm}[1]{\left\lVert#1\right\rVert}
%\providecommand{\norm}[1]{\lVert#1\rVert}
\providecommand{\mtx}[1]{\mathbf{#1}}
\providecommand{\mean}[1]{E\left[ #1 \right]}
\providecommand{\fourier}{\overset{\mathcal{F}}{ \rightleftharpoons}}
%\providecommand{\hilbert}{\overset{\mathcal{H}}{ \rightleftharpoons}}
\providecommand{\ztrans}{\overset{\mathcal{Z}}{ \rightleftharpoons}}
\providecommand{\system}{\overset{\mathcal{H}}{ \longleftrightarrow}}
	%\newcommand{\solution}[2]{\textbf{Solution:}{#1}}
\newcommand{\solution}{\noindent \textbf{Solution: }}
\newcommand{\cosec}{\,\text{cosec}\,}
\providecommand{\dec}[2]{\ensuremath{\overset{#1}{\underset{#2}{\gtrless}}}}
\newcommand{\myvec}[1]{\ensuremath{\begin{pmatrix}#1\end{pmatrix}}}
\newcommand{\mydet}[1]{\ensuremath{\begin{vmatrix}#1\end{vmatrix}}}
\providecommand{\gauss}[2]{\mathcal{N}\ensuremath{\left(#1,#2\right)}}
%\providecommand{\system}[1]{\overset{\mathcal{#1}}{ \longleftrightarrow}}
\newcommand*{\permcomb}[4][0mu]{{{}^{#3}\mkern#1#2_{#4}}}
\newcommand*{\perm}[1][-3mu]{\permcomb[#1]{P}}
\newcommand*{\comb}[1][-1mu]{\permcomb[#1]{C}}
%\numberwithin{equation}{section}
\numberwithin{equation}{subsection}
%\numberwithin{problem}{section}
%\numberwithin{definition}{section}
\makeatletter
\@addtoreset{figure}{problem}
\makeatother

\let\StandardTheFigure\thefigure
\let\vec\mathbf
%\renewcommand{\thefigure}{\theproblem.\arabic{figure}}
\renewcommand{\thefigure}{\theproblem}
%\setlist[enumerate,1]{before=\renewcommand\theequation{\theenumi.\arabic{equation}}
%\counterwithin{equation}{enumi}


%\renewcommand{\theequation}{\arabic{subsection}.\arabic{equation}}

\def\putbox#1#2#3{\makebox[0in][l]{\makebox[#1][l]{}\raisebox{\baselineskip}[0in][0in]{\raisebox{#2}[0in][0in]{#3}}}}
     \def\rightbox#1{\makebox[0in][r]{#1}}
     \def\centbox#1{\makebox[0in]{#1}}
     \def\topbox#1{\raisebox{-\baselineskip}[0in][0in]{#1}}
     \def\midbox#1{\raisebox{-0.5\baselineskip}[0in][0in]{#1}}

\vspace{3cm}

\title{
	\logo{
Probability and Random Variables
	}
}
\author{ G V V Sharma$^{*}$% <-this % stops a space
	\thanks{*The author is with the Department
		of Electrical Engineering, Indian Institute of Technology, Hyderabad
		502285 India e-mail:  gadepall@iith.ac.in. All content in this manual is released under GNU GPL.  Free and open source.}
	
}	
%\title{
%	\logo{Matrix Analysis through Octave}{\begin{center}\includegraphics[scale=.24]{tlc}\end{center}}{}{HAMDSP}
%}


% paper title
% can use linebreaks \\ within to get better formatting as desired
%\title{Matrix Analysis through Octave}
%
%
% author names and IEEE memberships
% note positions of commas and nonbreaking spaces ( ~ ) LaTeX will not break
% a structure at a ~ so this keeps an author's name from being broken across
% two lines.
% use \thanks{} to gain access to the first footnote area
% a separate \thanks must be used for each paragraph as LaTeX2e's \thanks
% was not built to handle multiple paragraphs
%

%\author{<-this % stops a space
%\thanks{}}
%}
% note the % following the last \IEEEmembership and also \thanks - 
% these prevent an unwanted space from occurring between the last author name
% and the end of the author line. i.e., if you had this:
% 
% \author{....lastname \thanks{...} \thanks{...} }
%                     ^------------^------------^----Do not want these spaces!
%
% a space would be appended to the last name and could cause every name on that
% line to be shifted left slightly. This is one of those "LaTeX things". For
% instance, "\textbf{A} \textbf{B}" will typeset as "A B" not "AB". To get
% "AB" then you have to do: "\textbf{A}\textbf{B}"
% \thanks is no different in this regard, so shield the last } of each \thanks
% that ends a line with a % and do not let a space in before the next \thanks.
% Spaces after \IEEEmembership other than the last one are OK (and needed) as
% you are supposed to have spaces between the names. For what it is worth,
% this is a minor point as most people would not even notice if the said evil
% space somehow managed to creep in.



% The paper headers
%\markboth{Journal of \LaTeX\ Class Files,~Vol.~6, No.~1, January~2007}%
%{Shell \MakeLowercase{\textit{et al.}}: Bare Demo of IEEEtran.cls for Journals}
% The only time the second header will appear is for the odd numbered pages
% after the title page when using the twoside option.
% 
% *** Note that you probably will NOT want to include the author's ***
% *** name in the headers of peer review papers.                   ***
% You can use \ifCLASSOPTIONpeerreview for conditional compilation here if
% you desire.




% If you want to put a publisher's ID mark on the page you can do it like
% this:
%\IEEEpubid{0000--0000/00\$00.00~\copyright~2007 IEEE}
% Remember, if you use this you must call \IEEEpubidadjcol in the second
% column for its text to clear the IEEEpubid mark.



% make the title area
\maketitle

\newpage

\tableofcontents

\bigskip

\renewcommand{\thefigure}{\theenumi}
\renewcommand{\thetable}{\theenumi}
%\renewcommand{\theequation}{\theenumi}

%\begin{abstract}
%%\boldmath
%In this letter, an algorithm for evaluating the exact analytical bit error rate  (BER)  for the piecewise linear (PL) combiner for  multiple relays is presented. Previous results were available only for upto three relays. The algorithm is unique in the sense that  the actual mathematical expressions, that are prohibitively large, need not be explicitly obtained. The diversity gain due to multiple relays is shown through plots of the analytical BER, well supported by simulations. 
%
%\end{abstract}
% IEEEtran.cls defaults to using nonbold math in the Abstract.
% This preserves the distinction between vectors and scalars. However,
% if the journal you are submitting to favors bold math in the abstract,
% then you can use LaTeX's standard command \boldmath at the very start
% of the abstract to achieve this. Many IEEE journals frown on math
% in the abstract anyway.

% Note that keywords are not normally used for peerreview papers.
%\begin{IEEEkeywords}
%Cooperative diversity, decode and forward, piecewise linear
%\end{IEEEkeywords}



% For peer review papers, you can put extra information on the cover
% page as needed:
% \ifCLASSOPTIONpeerreview
% \begin{center} \bfseries EDICS Category: 3-BBND \end{center}
% \fi
%
% For peerreview papers, this IEEEtran command inserts a page break and
% creates the second title. It will be ignored for other modes.
%\IEEEpeerreviewmaketitle

\begin{abstract}
This book provides a simple introduction to probability and random variables.   The contents are largely based on  NCERT textbooks from Class 9-12.
\end{abstract}


\section{Sum of Independent Random Variables}
\subsection{The Uniform Distribution}
Two dice, one blue and one grey, are thrown at the same time.   The event defined by the sum of the two numbers appearing on the top of the dice can have 11 possible outcomes 2, 3, 4, 5, 6, 6, 8, 9, 10, 11 and 12.  A student argues that each of these outcomes has a probability $\frac{1}{11}$.  Do you agree with this argument?  Justify your answer.

\renewcommand{\theequation}{\theenumi}
\renewcommand{\thefigure}{\theenumi}
\begin{enumerate}[label=\thesubsection.\arabic*.,ref=\thesubsection.\theenumi]
\numberwithin{equation}{enumi}
\numberwithin{figure}{enumi}

%\begin{abstract}
%We show that the problem of finding the probability of the sum of numbers appearing on top of two dice thrown at the same time can be solved used concepts in signal processing.  
%\begin{keywords}
%conditional probability, convolution, $Z$-transform
%\end{keywords}\bigskip
%
%
%\end{abstract}
%

\item  {\em The Uniform Distribution: }Let $X_i \in \cbrak{1,2,3,4,5,6}, i = 1,2,$ be the random variables representing the outcome for each die.  Assuming the dice to be fair, the probability mass function (pmf) is expressed as 
\begin{align}
\label{eq:dice_pmf_xi}
p_{X_i}(n) = \pr{X_i = n} = 
\begin{cases}
\frac{1}{6} & 1 \le n \le 6
\\
0 & otherwise
\end{cases}
\end{align}
The desired outcome is
\begin{align}
\label{eq:dice_xdef}
X &= X_1 + X_2,
\\
\implies X &\in \cbrak{1,2,\dots,12}
\end{align}
%
The objective is to show that
\begin{align}
p_X(n) \ne \frac{1}{11}
\label{eq:dice_wrong}
\end{align}
\item {\em Convolution: }
From \eqref{eq:dice_xdef},
\begin{align}
p_X(n) &= \pr{X_1 + X_2 = n} = \pr{X_1  = n -X_2}
\\
&= \sum_{k}^{}\pr{X_1  = n -k | X_2 = k}p_{X_2}(k)
\label{eq:dice_x_sum}
\end{align}
after unconditioning.  $\because X_1$ and $X_2$ are independent,
\begin{multline}
\pr{X_1  = n -k | X_2 = k} 
\\
= \pr{X_1  = n -k} = p_{X_1}(n-k)
\label{eq:dice_x1_indep}
\end{multline}
From \eqref{eq:dice_x_sum} and \eqref{eq:dice_x1_indep},
\begin{align}
p_X(n) = \sum_{k}^{}p_{X_1}(n-k)p_{X_2}(k) = p_{X_1}(n)*p_{X_2}(n)
\label{eq:dice_x_conv}
\end{align}
where $*$ denotes the convolution operation. 
%\cite{proakis_dsp}.  
Substituting from \eqref{eq:dice_pmf_xi}
in \eqref{eq:dice_x_conv},
\begin{align}
p_X(n) = \frac{1}{6}\sum_{k=1}^{6}p_{X_1}(n-k)= \frac{1}{6}\sum_{k=n-6}^{n-1}p_{X_1}(k)
\label{eq:dice_x_conv_x1}
\end{align}
\begin{align}
\because p_{X_1}(k) &= 0, \quad k \le 1, k \ge 6.
\end{align}
From \eqref{eq:dice_x_conv_x1},
%
\begin{align}
p_X(n) &= 
\begin{cases}
0 & n < 1
\\
\frac{1}{6}\sum_{k=1}^{n-1}p_{X_1}(k) &  1 \le n-1 \le  6
\\
\frac{1}{6}\sum_{k=n-6}^{6}p_{X_1}(k) & 1 < n-6 \le 6
\\
0 & n > 12
\end{cases}
\label{eq:dice_x_conv_cond}
\end{align}
Substituting from \eqref{eq:dice_pmf_xi} in \eqref{eq:dice_x_conv_cond},
\begin{align}
p_X(n) &= 
\begin{cases}
0 & n < 1
\\
\frac{n-1}{36} &  2 \le n \le  7
\\
\frac{13-n}{36} & 7 < n \le 12
\\
0 & n > 12
\end{cases}
\label{eq:dice_x_conv_final}
\end{align}
satisfying \eqref{eq:dice_wrong}.
\item {\em The $Z$-transform: }
The $Z$-transform of $p_X(n)$ is defined as 
%\cite{proakis_dsp}
\begin{align}
P_X(z) = \sum_{n = -\infty}^{\infty}p_X(n)z^{-n}, \quad z \in \mathbb{C}
\label{eq:dice_xz}
\end{align}
%
From \eqref{eq:dice_pmf_xi} and \eqref{eq:dice_xz}, 
\begin{align}
P_{X_1}(z) =P_{X_2}(z) &= \frac{1}{6}\sum_{n = 1}^{6}z^{-n}
\\
&=\frac{z^{-1}\brak{1-z^{-6}}}{6\brak{1-z^{-1}}}, \quad \abs{z} > 1
\label{eq:dice_xiz}
\end{align}
upon summing up the geometric progression.  
\begin{align}
\because p_X(n) &= p_{X_1}(n)*p_{X_2}(n),
\\
P_X(z) &= P_{X_1}(z)P_{X_2}(z)
\label{eq:dice_xzprod_def}
\end{align}
The above property follows from Fourier analysis and is fundamental to signal processing. 
%\cite{proakis_dsp}. 
From \eqref{eq:dice_xiz} and \eqref{eq:dice_xzprod_def},
\begin{align}
P_X(z) &= \cbrak{\frac{z^{-1}\brak{1-z^{-6}}}{6\brak{1-z^{-1}}}}^2
\\
&= \frac{1}{36}\frac{z^{-2}\brak{1-2z^{-6}+z^{-12}}}{\brak{1-z^{-1}}^2}
\label{eq:dice_xzprod}
\end{align}
Using the fact that 
%\cite{proakis_dsp}
\begin{align}
p_X(n-k) &\system{Z}P_X(z)z^{-k},
\\
nu(n)&\system{Z} \frac{z^{-1}}{\brak{1-z^{-1}}^2}
\end{align}
after some algebra, it can be shown that
%{\tiny
\begin{multline}
\frac{1}{36}\lsbrak{\brak{n-1}u(n-1) - 2 \brak{n-7}u(n-7)}
\\
\rsbrak{ +\brak{n-13}u(n-13)}
\\
\system{Z}
\frac{1}{36}\frac{z^{-2}\brak{1-2z^{-6}+z^{-12}}}{\brak{1-z^{-1}}^2}
\label{eq:dice_xz_closed}
\end{multline}
%}

where 
\begin{align}
u(n) =
\begin{cases}
1 & n \ge 0
\\
0 & n < 0
\end{cases}
\end{align}

From \eqref{eq:dice_xz}, \eqref{eq:dice_xzprod} and \eqref{eq:dice_xz_closed}
\begin{multline}
p_{X}(n) = \frac{1}{36}\lsbrak{\brak{n-1}u(n-1) 
}
\\
\rsbrak{- 2 \brak{n-7}u(n-7)+\brak{n-13}u(n-13)}
\end{multline}
which is the same as \eqref{eq:dice_x_conv_final}.  Note that  \eqref{eq:dice_x_conv_final} can be obtained from \eqref{eq:dice_xz_closed} using contour integration as well.
% \cite{proakis_dsp}.  

\item 
The experiment of rolling the dice was simulated using Python for 10000 samples.  These were generated using Python libraries for uniform distribution. The frequencies for each outcome were then used to compute the resulting pmf, which  is plotted in Figure \ref{fig:dice}.  The theoretical pmf obtained in \eqref{eq:dice_x_conv_final} is plotted for comparison.  
%
\begin{figure}[!ht]
\centering
\includegraphics[width=\columnwidth]{./figs/sum/pmf.png}
\caption{Plot of $p_X(n)$.  Simulations are close to the analysis. }
\label{fig:dice}
\end{figure}
\item The python code is available in 
\begin{lstlisting}
/codes/sum/dice.py
\end{lstlisting}

%\item 
%We have shown how a simple problem of throwing a dice can be used for learning not just probability but concepts in  signal processing as well.  Inversion of the $Z$-transform for finding the pmf using contour integration, though not discussed here, opens a window to complex analysis too.  Note that the solutions that are provided  can be easily understood using high school math like arithmetic and geometric sums.  Thus, school students can use a bit of college level math to obtain simpler solutions to their problems.  College students can be exposed to advanced mathematics through simple problems from high school texts.  In addition, both can learn how to verify theoretical results through computer simulations.

%\bibliographystyle{tMES}
%\bibliography{school}
%
%\end{document}
\end{enumerate}

\section{Cumulative Distribution Function}
\subsection{The Bernoulli Distribution}
\renewcommand{\theequation}{\theenumi}
\renewcommand{\thefigure}{\theenumi}
\begin{enumerate}[label=\thesubsection.\arabic*.,ref=\thesubsection.\theenumi]
\numberwithin{equation}{enumi}
\numberwithin{figure}{enumi}
\numberwithin{table}{enumi}
%
\item Find the probability of getting a head when a coin is tossed once. Also
find the probability of getting a tail.
\\
\solution  
Let the random variable be $X\in \cbrak{0,1}$.  Then
\begin{align}
\pr{X=0}=\pr{X=1}=\frac{1}{2}
\end{align}
The following code simulates the event for 100 coin tosses
\begin{lstlisting}
codes/bernoulli/coin.py
\end{lstlisting}
\item {\em Binomial Distribution: }  In general the binomial distribution is defined using the PMF
\begin{align}
p_{X}\brak{n} 
= 
\begin{cases}
p & n = 0
\\
1-p & n = 1
\\
\text{otherwise}
\end{cases}
\end{align}

\item A jar contains 24 marbles, some are green and others are blue. If a marble is drawn at random from the jar, the probability that it is green is
$\frac{2}{3}$. Find the number of blue balls (marbles) in the jar.\\
\solution Let random variable $X\in\{0,1\}$ denote the outcomes of the experiment of drawing a marble from a jar as shown in Table \ref{table:bernoulli}
%
\begin{table}
\centering
\caption{}
\renewcommand{\theequation}{\theenumi}
\renewcommand{\thefigure}{\theenumi}
\begin{enumerate}[label=\thesubsection.\arabic*.,ref=\thesubsection.\theenumi]
\numberwithin{equation}{enumi}
\numberwithin{figure}{enumi}
\numberwithin{table}{enumi}
%
\item Find the probability of getting a head when a coin is tossed once. Also
find the probability of getting a tail.
\\
\solution  
Let the random variable be $X\in \cbrak{0,1}$.  Then
\begin{align}
\pr{X=0}=\pr{X=1}=\frac{1}{2}
\end{align}
The following code simulates the event for 100 coin tosses
\begin{lstlisting}
codes/bernoulli/coin.py
\end{lstlisting}
\item {\em Binomial Distribution: }  In general the binomial distribution is defined using the PMF
\begin{align}
p_{X}\brak{n} 
= 
\begin{cases}
p & n = 0
\\
1-p & n = 1
\\
\text{otherwise}
\end{cases}
\end{align}

\item A jar contains 24 marbles, some are green and others are blue. If a marble is drawn at random from the jar, the probability that it is green is
$\frac{2}{3}$. Find the number of blue balls (marbles) in the jar.\\
\solution Let random variable $X\in\{0,1\}$ denote the outcomes of the experiment of drawing a marble from a jar as shown in Table \ref{table:bernoulli}
%
\begin{table}
\centering
\caption{}
\renewcommand{\theequation}{\theenumi}
\renewcommand{\thefigure}{\theenumi}
\begin{enumerate}[label=\thesubsection.\arabic*.,ref=\thesubsection.\theenumi]
\numberwithin{equation}{enumi}
\numberwithin{figure}{enumi}
\numberwithin{table}{enumi}
%
\item Find the probability of getting a head when a coin is tossed once. Also
find the probability of getting a tail.
\\
\solution  
Let the random variable be $X\in \cbrak{0,1}$.  Then
\begin{align}
\pr{X=0}=\pr{X=1}=\frac{1}{2}
\end{align}
The following code simulates the event for 100 coin tosses
\begin{lstlisting}
codes/bernoulli/coin.py
\end{lstlisting}
\item {\em Binomial Distribution: }  In general the binomial distribution is defined using the PMF
\begin{align}
p_{X}\brak{n} 
= 
\begin{cases}
p & n = 0
\\
1-p & n = 1
\\
\text{otherwise}
\end{cases}
\end{align}

\item A jar contains 24 marbles, some are green and others are blue. If a marble is drawn at random from the jar, the probability that it is green is
$\frac{2}{3}$. Find the number of blue balls (marbles) in the jar.\\
\solution Let random variable $X\in\{0,1\}$ denote the outcomes of the experiment of drawing a marble from a jar as shown in Table \ref{table:bernoulli}
%
\begin{table}
\centering
\caption{}
\input{./tables/bernoulli/bernoulli.tex}
\label{table:bernoulli}
\end{table}
%
From the given information,
\begin{align}
\label{eq:bernoulli_x=1}
p_X(1)&=\frac{2}{3}
\\
\implies p = 1 -  p_X(1) &= \frac{1}{3}
\\
n\brak{X=0}+
n\brak{X=1} &= 24
\label{eq:bernoulli_sum}
\end{align}
%
%Thus, from \eqref{eq:bernoulli_x=1} 
%\begin{align}
% P(X=0)&=1-P(X=1)\nonumber\\
% &=1-\frac{2}{3}=\frac{1}{3}.
%\end{align}
%
%Since, we know 
$\because$
\begin{equation}
p=\frac{n\brak{X=0} }{n\brak{X=0}+
n\brak{X=1}},
 \label{eq:blue_marb}
\end{equation} 
from \eqref{eq:blue_marb} and \eqref{eq:bernoulli_sum},
%So, using Eq.~\eqref{eq:blue_marb} we have, 
\begin{align}
n\brak{X=0}  &=p\cbrak{n\brak{X=0}+n\brak{X=1}}
\\
&=\frac{1}{3}\times 24=8.
\label{eq:green_marble}
\end{align}
% 
The following code generates the number of blue marbles 
\begin{lstlisting}
codes/bernoulli/bernoulli.py
\end{lstlisting}


\end{enumerate}



\label{table:bernoulli}
\end{table}
%
From the given information,
\begin{align}
\label{eq:bernoulli_x=1}
p_X(1)&=\frac{2}{3}
\\
\implies p = 1 -  p_X(1) &= \frac{1}{3}
\\
n\brak{X=0}+
n\brak{X=1} &= 24
\label{eq:bernoulli_sum}
\end{align}
%
%Thus, from \eqref{eq:bernoulli_x=1} 
%\begin{align}
% P(X=0)&=1-P(X=1)\nonumber\\
% &=1-\frac{2}{3}=\frac{1}{3}.
%\end{align}
%
%Since, we know 
$\because$
\begin{equation}
p=\frac{n\brak{X=0} }{n\brak{X=0}+
n\brak{X=1}},
 \label{eq:blue_marb}
\end{equation} 
from \eqref{eq:blue_marb} and \eqref{eq:bernoulli_sum},
%So, using Eq.~\eqref{eq:blue_marb} we have, 
\begin{align}
n\brak{X=0}  &=p\cbrak{n\brak{X=0}+n\brak{X=1}}
\\
&=\frac{1}{3}\times 24=8.
\label{eq:green_marble}
\end{align}
% 
The following code generates the number of blue marbles 
\begin{lstlisting}
codes/bernoulli/bernoulli.py
\end{lstlisting}


\end{enumerate}



\label{table:bernoulli}
\end{table}
%
From the given information,
\begin{align}
\label{eq:bernoulli_x=1}
p_X(1)&=\frac{2}{3}
\\
\implies p = 1 -  p_X(1) &= \frac{1}{3}
\\
n\brak{X=0}+
n\brak{X=1} &= 24
\label{eq:bernoulli_sum}
\end{align}
%
%Thus, from \eqref{eq:bernoulli_x=1} 
%\begin{align}
% P(X=0)&=1-P(X=1)\nonumber\\
% &=1-\frac{2}{3}=\frac{1}{3}.
%\end{align}
%
%Since, we know 
$\because$
\begin{equation}
p=\frac{n\brak{X=0} }{n\brak{X=0}+
n\brak{X=1}},
 \label{eq:blue_marb}
\end{equation} 
from \eqref{eq:blue_marb} and \eqref{eq:bernoulli_sum},
%So, using Eq.~\eqref{eq:blue_marb} we have, 
\begin{align}
n\brak{X=0}  &=p\cbrak{n\brak{X=0}+n\brak{X=1}}
\\
&=\frac{1}{3}\times 24=8.
\label{eq:green_marble}
\end{align}
% 
The following code generates the number of blue marbles 
\begin{lstlisting}
codes/bernoulli/bernoulli.py
\end{lstlisting}


\end{enumerate}



\subsection{The Binomial Distribution}
In a hurdle race, a player has to cross 10 hurdles. The probability that he will
clear each hurdle is $\frac{5}{6}$. What is the probability that he will knock down fewer than 2 hurdles?
\renewcommand{\theequation}{\theenumi}
\renewcommand{\thefigure}{\theenumi}
\begin{enumerate}[label=\thesubsection.\arabic*.,ref=\thesubsection.\theenumi]
\numberwithin{equation}{enumi}
\numberwithin{figure}{enumi}
\numberwithin{table}{enumi}
%
%
\item  Let $X_i \in \cbrak{0,1}$ represent the $ith$ hurdle where $1$ denotes a hurdle being knocked down.  Then, $X_i$ has a bernoulli distribution with parameter
\begin{align}
p = 1-\frac{5}{6} = \frac{1}{6}
\label{eq:p_binom_exam}
\end{align}

\item {\em The Binomial Distribution: } Let
\begin{align}
\label{eq:bern_binom}
X = \sum_{i=1}^{n}X_i
\end{align}
%
where $n$ is the total number of hurdles.
Then $X$ has a binomial distribution.  Then, for 
\begin{align}
p_{X_i}(n) \ztrans P_{X_i}(z),
\end{align}
yielding
\begin{align}
 P_{X_i}(z) = 1-p + pz^{-1}
\end{align}
 with
%
Using the fact that $X_i$ are i.i.d.,
\begin{align}
\label{eq:ztrans_binom}
 P_{X}(z) &= \brak{1-p + pz^{-1}}^n
\\
&= \sum_{k=0}^{n}\comb{n}{k}p^{k}\brak{1-p}^{n-k}z^{-k}
\\
\implies p_{X}(k) &= 
\begin{cases}
\comb{n}{k}p^{n-k}\brak{1-p}^k & 0 \le k \le n
\\
0 & \text{otherwise}
\end{cases}
\label{eq:pdf_binom}
\end{align}
%
The cumulative distribution function of $X$ is defined as
\begin{align}
F_{X}(r) = \pr{X\le r} = \sum_{k=0}^{r}\comb{n}{k}p^{k}\brak{1-p}^{n-k}
\label{eq:cdf_binom}
\end{align}
%
upon substituting from \eqref{eq:pdf_binom}.

\item {\em Evaluationg the Probability: }Substituting from \eqref{eq:p_binom_exam} in \eqref{eq:cdf_binom},

\begin {align}
\pr{X < 2} &= F_{X}(1)
\\
&=\sum_{k=0}^{1}\comb{n}{k}\brak{\frac{5}{6}}^{10-k}\brak{\frac{1}{6}}^k
\\
&=3\brak{\frac{5}{6}}^{10} = 0.4845167486695371 
\label{eq:bern_binom_ans}
\end{align}
%
which is the desired probability.
%
\item The following code verifies the above result.
\begin{lstlisting}
codes/binomial/binomial.py
\end{lstlisting}
\end{enumerate}

\section{Central Limit Theorem: Gaussian Distribution}
\renewcommand{\thefigure}{\theenumi}
\renewcommand{\thetable}{\theenumi}
%%
%\begin{enumerate}[label=\thesection.\arabic*
%,ref=\thesection.\theenumi]

\subsection{Bernoulli to Gaussian}
\begin{enumerate}[label=\thesubsection.\arabic*
,ref=\thesection.\theenumi]


\item {\em Mean :}  The mean of the bernoulli distribution is 
\begin{align}
\mu = E\brak{X_i}  = \sum_{k=0}^{1}kp_{X_i}(k) = p = \frac{1}{6}
\end{align}
\item {\em Moment:}  The moment of the distribution is defined as
\begin{align}
E\brak{X_i^r}  = \sum_{k=0}^{1}k^rp_{X_i}(k) = p = \frac{1}{6}
\end{align}

%The second moment of the bernoulli distribution is 
%\begin{align}
%E\brak{X_i}  = \sum_{k=0}^{1}kp_{X_i}(k) = p = \frac{1}{6}
%\end{align}
\item {\em Variance :}  The variance of the bernoulli distribution is defined as
\begin{align}
\sigma^2 &= E\brak{X-E\brak{X}}^2  = E\brak{X^2}-E^2\brak{X} 
\\
&=p-p^2 = p\brak{1-p} = \frac{5}{36}
\end{align}
%
The standard deviation 
\begin{align}
\sigma =  \sqrt{p\brak{1-p}}
\end{align}
%
\item {\em The Gaussian Distribution: }  Define
\begin{align}
\label{eq:bern_gauss}
G = \frac{1}{\sqrt{n}}\sum_{k=1}^{n}\frac{X_i-\mu}{\sigma}
\end{align}
%
\item {\em Approximating Binomial Using Gaussian: } From \eqref{eq:bern_gauss}
and \eqref{eq:bern_binom},
%
\begin{align}
X & \approx \sigma\sqrt{n}G + n\mu 
\\
\implies F_X(k) &= \pr{\sigma\sqrt{n}G + n\mu  \le k }
\\
 &= F_G\brak{\frac{k-n\mu}{\sigma\sqrt{n}}} \approx \phi\brak{\frac{k-n\mu}{\sigma\sqrt{n}}} 
\label{eq:bern_gaussian_cdf}
\end{align}
where 
\begin{align}
\phi_{X}(x) = \int^{x}_{-\infty} \frac{1}{\sqrt{2\pi}}e^{-\frac{x^2}{2}}, -\infty < x < \infty
\end{align}
\item The 
probability density function (PDF) 
of $G$ is
%
\begin{align}
p_{G}(x) &= \frac{d}{dx}F_{X}(x)
\\
 &=  \frac{1}{{\sigma\sqrt{n}}}\phi^{\prime}\brak{\frac{k-n\mu}{\sigma\sqrt{n}}} 
\label{eq:bern_gaussian_pdf}
\end{align}
%
For large $n$, $G$ is a continuous distribution with probability density function (PDF)
\begin{align}
p_G(x) =  \frac{1}{\sqrt{2\pi}}\exp\brak{-\frac{x^2}{2}}, -\infty < x < \infty,
\end{align}
%
\item {\em Evaluationg the Probability: }  From \ref{eq:bern_gaussian_cdf}
and \ref{eq:bern_gaussian_pdf},
\begin{align}
\pr{X \le 1 } &= F_{G}(1) = p_G(0)+p_G(1) 
\\
&\approx 
0.41299463887797094
\label{eq:bern_gauss_ans}
\end{align}
which is close to \eqref{eq:bern_binom_ans}.
%
\end{enumerate}
\subsection{Uniform to Gaussian}
\begin{enumerate}[label=\thesubsection.\arabic*
,ref=\thesection.\theenumi]

\item
Generate $10^6$ samples of the random variable
%
\begin{equation}
X = \sum_{i=1}^{12}U_i -6
\end{equation}
%
using a C program, where $U_i, i = 1,2,\dots, 12$ are  a set of independent uniform random variables between 0 and 1
and save in a file called gau.dat
\\
\solution Download the following files and execute the  C program.
\begin{lstlisting}
codes/cdf/exrand.c
codes/cdf/coeffs.h
\end{lstlisting}

%
\item
Load gau.dat in python and plot the empirical CDF of $X$ using the samples in gau.dat. What properties does a CDF have?
\\
\solution The CDF of $X$ is plotted in Fig. \ref{fig:gauss_cdf}
\begin{figure}
\centering
\includegraphics[width=\columnwidth]{./figs/clt/gauss_cdf}
\caption{The CDF of $X$}
\label{fig:gauss_cdf}
\end{figure}


\item
Load gau.dat in python and plot the empirical PDF of $X$ using the samples in gau.dat. The PDF of $X$ is defined as
\begin{align}
p_{X}(x) = \frac{d}{dx}F_{X}(x)
\end{align}
What properties does the PDF have?
\\
\solution The PDF of $X$ is plotted in Fig. \ref{fig:gauss_pdf} using the code below
\begin{lstlisting}
codes/clt/pdf_plot.py
\end{lstlisting}

\begin{figure}
\centering
\includegraphics[width=\columnwidth]{./figs/clt/gauss_pdf}
\caption{The PDF of $X$}
\label{fig:gauss_pdf}
\end{figure}

\item Find the mean and variance of $X$ by writing a C program.
\item Given that 
\begin{align}
p_{X}(x) = \frac{1}{\sqrt{2\pi}}\exp\brak{-\frac{x^2}{2}}, -\infty < x < \infty,
\end{align}
repeat the above exercise theoretically.
%
Let $U$ be a uniform random variable between 0 and 1.
%\begin{enumerate}[label=\thesection.\arabic*
%,ref=\thesection.\theenumi]

%
\item
Load the uni.dat file into python and plot the empirical CDF of $U$ using the samples in uni.dat. The CDF is defined as
\begin{align}
F_{U}(x) = \pr{U \le x}
\end{align}
\\
\solution  The following code plots Fig. \ref{fig:uni_cdf}
\begin{lstlisting}
codes/cdf/cdf_plot.py
\end{lstlisting}
\begin{figure}
\centering
\includegraphics[width=\columnwidth]{./figs/cdf/uni_cdf}
\caption{The CDF of $U$}
\label{fig:uni_cdf}
\end{figure}

%\item Generate $10^6$ samples of $U$ using a C program and save into a file called uni.dat .
%\\


%
\item
Find a  theoretical expression for $F_{U}(x)$.

\item
The mean of $U$ is defined as
%
\begin{equation}
E\sbrak{U} = \frac{1}{N}\sum_{i=1}^{N}U_i
\end{equation}
%
and its variance as
%
\begin{equation}
\text{var}\sbrak{U} = E\sbrak{U- E\sbrak{U}}^2 
\end{equation}

Write a C program to  find the mean and variance of $U$. 
\item Verify your result theoretically given that
%
\begin{equation}
E\sbrak{U^k} = \int_{-\infty}^{\infty}x^kdF_{U}(x)
\end{equation}

\end{enumerate}




\section{Cumulative Distribution Function}
\renewcommand{\thefigure}{\theenumi}
\renewcommand{\thetable}{\theenumi}
%%
Let $U$ be a uniform random variable between 0 and 1.
%\begin{enumerate}[label=\thesection.\arabic*
%,ref=\thesection.\theenumi]
\item Generate $10^6$ samples of $U$ using a C program and save into a file called uni.dat .
\\
\solution Download the following files and execute the  C program.
\begin{lstlisting}
codes/cdf/exrand.c
codes/cdf/coeffs.h
\end{lstlisting}

%
\item
Load the uni.dat file into python and plot the empirical CDF of $U$ using the samples in uni.dat. The CDF is defined as
\begin{align}
F_{U}(x) = \pr{U \le x}
\end{align}
\\
\solution  The following code plots Fig. \ref{fig:uni_cdf}
\begin{lstlisting}
codes/cdf/cdf_plot.py
\end{lstlisting}
\begin{figure}
\centering
\includegraphics[width=\columnwidth]{./figs/cdf/uni_cdf}
\caption{The CDF of $U$}
\label{fig:uni_cdf}
\end{figure}

%
\item
Find a  theoretical expression for $F_{U}(x)$.

\item
The mean of $U$ is defined as
%
\begin{equation}
E\sbrak{U} = \frac{1}{N}\sum_{i=1}^{N}U_i
\end{equation}
%
and its variance as
%
\begin{equation}
\text{var}\sbrak{U} = E\sbrak{U- E\sbrak{U}}^2 
\end{equation}

Write a C program to  find the mean and variance of $U$. 
\item Verify your result theoretically given that
%
\begin{equation}
E\sbrak{U^k} = \int_{-\infty}^{\infty}x^kdF_{U}(x)
\end{equation}
%
\end{enumerate}





%\section{Bernoulli Distribution}
%\subsection{Distance from a plane to a point}
%\renewcommand{\theequation}{\theenumi}
\renewcommand{\thefigure}{\theenumi}
\begin{enumerate}[label=\thesubsection.\arabic*.,ref=\thesubsection.\theenumi]
\numberwithin{equation}{enumi}
\numberwithin{figure}{enumi}
\numberwithin{table}{enumi}
%
\item Find the probability of getting a head when a coin is tossed once. Also
find the probability of getting a tail.
\\
\solution  
Let the random variable be $X\in \cbrak{0,1}$.  Then
\begin{align}
\pr{X=0}=\pr{X=1}=\frac{1}{2}
\end{align}
The following code simulates the event for 100 coin tosses
\begin{lstlisting}
codes/bernoulli/coin.py
\end{lstlisting}
\item {\em Binomial Distribution: }  In general the binomial distribution is defined using the PMF
\begin{align}
p_{X}\brak{n} 
= 
\begin{cases}
p & n = 0
\\
1-p & n = 1
\\
\text{otherwise}
\end{cases}
\end{align}

\item A jar contains 24 marbles, some are green and others are blue. If a marble is drawn at random from the jar, the probability that it is green is
$\frac{2}{3}$. Find the number of blue balls (marbles) in the jar.\\
\solution Let random variable $X\in\{0,1\}$ denote the outcomes of the experiment of drawing a marble from a jar as shown in Table \ref{table:bernoulli}
%
\begin{table}
\centering
\caption{}
\renewcommand{\theequation}{\theenumi}
\renewcommand{\thefigure}{\theenumi}
\begin{enumerate}[label=\thesubsection.\arabic*.,ref=\thesubsection.\theenumi]
\numberwithin{equation}{enumi}
\numberwithin{figure}{enumi}
\numberwithin{table}{enumi}
%
\item Find the probability of getting a head when a coin is tossed once. Also
find the probability of getting a tail.
\\
\solution  
Let the random variable be $X\in \cbrak{0,1}$.  Then
\begin{align}
\pr{X=0}=\pr{X=1}=\frac{1}{2}
\end{align}
The following code simulates the event for 100 coin tosses
\begin{lstlisting}
codes/bernoulli/coin.py
\end{lstlisting}
\item {\em Binomial Distribution: }  In general the binomial distribution is defined using the PMF
\begin{align}
p_{X}\brak{n} 
= 
\begin{cases}
p & n = 0
\\
1-p & n = 1
\\
\text{otherwise}
\end{cases}
\end{align}

\item A jar contains 24 marbles, some are green and others are blue. If a marble is drawn at random from the jar, the probability that it is green is
$\frac{2}{3}$. Find the number of blue balls (marbles) in the jar.\\
\solution Let random variable $X\in\{0,1\}$ denote the outcomes of the experiment of drawing a marble from a jar as shown in Table \ref{table:bernoulli}
%
\begin{table}
\centering
\caption{}
\renewcommand{\theequation}{\theenumi}
\renewcommand{\thefigure}{\theenumi}
\begin{enumerate}[label=\thesubsection.\arabic*.,ref=\thesubsection.\theenumi]
\numberwithin{equation}{enumi}
\numberwithin{figure}{enumi}
\numberwithin{table}{enumi}
%
\item Find the probability of getting a head when a coin is tossed once. Also
find the probability of getting a tail.
\\
\solution  
Let the random variable be $X\in \cbrak{0,1}$.  Then
\begin{align}
\pr{X=0}=\pr{X=1}=\frac{1}{2}
\end{align}
The following code simulates the event for 100 coin tosses
\begin{lstlisting}
codes/bernoulli/coin.py
\end{lstlisting}
\item {\em Binomial Distribution: }  In general the binomial distribution is defined using the PMF
\begin{align}
p_{X}\brak{n} 
= 
\begin{cases}
p & n = 0
\\
1-p & n = 1
\\
\text{otherwise}
\end{cases}
\end{align}

\item A jar contains 24 marbles, some are green and others are blue. If a marble is drawn at random from the jar, the probability that it is green is
$\frac{2}{3}$. Find the number of blue balls (marbles) in the jar.\\
\solution Let random variable $X\in\{0,1\}$ denote the outcomes of the experiment of drawing a marble from a jar as shown in Table \ref{table:bernoulli}
%
\begin{table}
\centering
\caption{}
\input{./tables/bernoulli/bernoulli.tex}
\label{table:bernoulli}
\end{table}
%
From the given information,
\begin{align}
\label{eq:bernoulli_x=1}
p_X(1)&=\frac{2}{3}
\\
\implies p = 1 -  p_X(1) &= \frac{1}{3}
\\
n\brak{X=0}+
n\brak{X=1} &= 24
\label{eq:bernoulli_sum}
\end{align}
%
%Thus, from \eqref{eq:bernoulli_x=1} 
%\begin{align}
% P(X=0)&=1-P(X=1)\nonumber\\
% &=1-\frac{2}{3}=\frac{1}{3}.
%\end{align}
%
%Since, we know 
$\because$
\begin{equation}
p=\frac{n\brak{X=0} }{n\brak{X=0}+
n\brak{X=1}},
 \label{eq:blue_marb}
\end{equation} 
from \eqref{eq:blue_marb} and \eqref{eq:bernoulli_sum},
%So, using Eq.~\eqref{eq:blue_marb} we have, 
\begin{align}
n\brak{X=0}  &=p\cbrak{n\brak{X=0}+n\brak{X=1}}
\\
&=\frac{1}{3}\times 24=8.
\label{eq:green_marble}
\end{align}
% 
The following code generates the number of blue marbles 
\begin{lstlisting}
codes/bernoulli/bernoulli.py
\end{lstlisting}


\end{enumerate}



\label{table:bernoulli}
\end{table}
%
From the given information,
\begin{align}
\label{eq:bernoulli_x=1}
p_X(1)&=\frac{2}{3}
\\
\implies p = 1 -  p_X(1) &= \frac{1}{3}
\\
n\brak{X=0}+
n\brak{X=1} &= 24
\label{eq:bernoulli_sum}
\end{align}
%
%Thus, from \eqref{eq:bernoulli_x=1} 
%\begin{align}
% P(X=0)&=1-P(X=1)\nonumber\\
% &=1-\frac{2}{3}=\frac{1}{3}.
%\end{align}
%
%Since, we know 
$\because$
\begin{equation}
p=\frac{n\brak{X=0} }{n\brak{X=0}+
n\brak{X=1}},
 \label{eq:blue_marb}
\end{equation} 
from \eqref{eq:blue_marb} and \eqref{eq:bernoulli_sum},
%So, using Eq.~\eqref{eq:blue_marb} we have, 
\begin{align}
n\brak{X=0}  &=p\cbrak{n\brak{X=0}+n\brak{X=1}}
\\
&=\frac{1}{3}\times 24=8.
\label{eq:green_marble}
\end{align}
% 
The following code generates the number of blue marbles 
\begin{lstlisting}
codes/bernoulli/bernoulli.py
\end{lstlisting}


\end{enumerate}



\label{table:bernoulli}
\end{table}
%
From the given information,
\begin{align}
\label{eq:bernoulli_x=1}
p_X(1)&=\frac{2}{3}
\\
\implies p = 1 -  p_X(1) &= \frac{1}{3}
\\
n\brak{X=0}+
n\brak{X=1} &= 24
\label{eq:bernoulli_sum}
\end{align}
%
%Thus, from \eqref{eq:bernoulli_x=1} 
%\begin{align}
% P(X=0)&=1-P(X=1)\nonumber\\
% &=1-\frac{2}{3}=\frac{1}{3}.
%\end{align}
%
%Since, we know 
$\because$
\begin{equation}
p=\frac{n\brak{X=0} }{n\brak{X=0}+
n\brak{X=1}},
 \label{eq:blue_marb}
\end{equation} 
from \eqref{eq:blue_marb} and \eqref{eq:bernoulli_sum},
%So, using Eq.~\eqref{eq:blue_marb} we have, 
\begin{align}
n\brak{X=0}  &=p\cbrak{n\brak{X=0}+n\brak{X=1}}
\\
&=\frac{1}{3}\times 24=8.
\label{eq:green_marble}
\end{align}
% 
The following code generates the number of blue marbles 
\begin{lstlisting}
codes/bernoulli/bernoulli.py
\end{lstlisting}


\end{enumerate}



\section{Stochastic Geometry}
\input{./chapters/stochastic.tex}
\section{Transformation of Variables}
\input{./chapters/trans.tex}
%
\section{Conditional Probability}
\input{./chapters/cond.tex}
%%
\section{Two Dimensions}
\input{./chapters/twoD.tex}
%%
\section{Transform Domain}
\input{./chapters/laplace.tex}
%%
\section{Uniform to Other}
\input{./chapters/uni2other.tex}

%\section{Sum of i.i.d random variables}
%\input{./chapters/conv.tex}


\end{document}


